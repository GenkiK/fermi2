\chapter{緒言}

% 多電子原子の発光線は非常に多く存在し複雑である.
% 各発光線の波長間隔が狭いので,発光線を分離させて観測するためには高い分解能の分光器が必要である.
代表的な分光器の配置であるツェルニ・ターナ型分光器では入射した光を平行にするためのコリメータと光を結像させる結像器に凹面鏡を使用している.
この際,回折格子が他の光路を遮らない配置にするため,凹面鏡で光を軸外反射して光軸を曲げる必要がある.
光が軸外反射をすることで球面収差やコマ収差などの収差が発生するため,分光器で実現できる分解能が制限される.
% そこで本研究ではコリメータ及び結像器にアクロマティックレンズを用い,光学系を共軸に配置することで収差を抑えた分光器を開発した.
それに対して,コリメータ及び結像器にレンズを用いると光学系を共軸に配置することができるので収差を抑えた分光器を実現することができる.
先行研究では焦点距離が300 mmのレンズを使って装置幅が14 pmの分光器が開発されている\cite{senkou}.
本研究では焦点距離が1525 mmと長焦点のレンズを用い高波長分散にすることで,高分解能な分光器を開発した.





% ただし,レンズには色収差があり波長ごとに焦点距離が変化してしまう.
% そのため,レンズ下に一軸自動ステージを付け焦点の調整を自動化した.

