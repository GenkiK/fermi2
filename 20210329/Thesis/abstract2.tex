\begin{abstract}

% 多くの分光器では光路内に凹面鏡による軸外反射を含み,収差が大きくなる.
本研究ではコリメータ及び結像器に焦点距離が1525 mmのアクロマティックレンズを使用することで,光学系を共軸に配置した回折格子分光器を開発した.
レンズには色収差があり,波長によって焦点距離が変化する.
そこで観測する光の波長に対応する回折格子の角度及び検出器受光面に焦点が合うレンズの位置の対応関係を求めた.
Pythonスクリプトを用いることで求めた対応関係を満たすようにレンズの位置及び回折格子の角度を自動制御可能にした.
完成した分光器の装置幅は6.9~9.1 pmとなり,光線追跡によるシミュレーションによって求めた理論装置幅の約2.3~3.0倍となった.
% ただし,レンズには色収差があり波長ごとに焦点距離が変化する.
% そこで観測する光の波長に対応する回折格子の角度及び最適なレンズの位置の対応関係を求めた.
% さら分光計測を自動化するため,Pythonスクリプトによって自動制御可能にした.
% 完成した分光器の装置幅は$6.9\times10^{-3}$~$9.1\times10^{-3}$ nmとなり,
% 光線追跡によって求めたレンズ焦点でのスポットサイズ,回折限界,スリットの幅から求めた理論装置幅の約2.3~3.0倍となった.
% 完成した分光器の装置幅は$6.9\times10^{-3}$~$9.1\times10^{-3}$ nmほどとなり,分光器の理論分解能の約2.3~3.0倍となった.
\end{abstract}